    \documentclass[12pt,a4paper]{article}

% Pacotes para o português.
\usepackage[portuguese, provide=*]{babel}
\usepackage{graphicx}
\usepackage{listings}
\usepackage{xcolor}
\usepackage{longtable}
\usepackage{indentfirst}
\usepackage{url}
\usepackage{array}
\usepackage[top=2.5cm, bottom=2.5cm, left=2.5cm, right=2.5cm]{geometry}
\usepackage{multirow}
\usepackage{amssymb}
\usepackage{amsmath}
\usepackage{caption}
\usepackage{setspace}
\usepackage{breakcites}
\usepackage{float}
\usepackage{lipsum}
\usepackage{booktabs}
\usepackage{fancyhdr}
\usepackage{hyperref}
\usepackage{amsmath}
\usepackage{amssymb}
\usepackage{listings}
\usepackage{xcolor}
\usepackage{microtype}
\usepackage{subcaption}

\definecolor{codegreen}{rgb}{0,0.6,0}
\definecolor{codegray}{rgb}{0.5,0.5,0.5}
\definecolor{codepurple}{rgb}{0.58,0,0.82}
\definecolor{backcolor}{rgb}{0.95,0.95,0.92}
\lstdefinestyle{mystyle}{
    backgroundcolor=\color{backcolor},   
    commentstyle=\color{codegreen},
    keywordstyle=\color{magenta},
    numberstyle=\tiny\color{codegray},
    stringstyle=\color{codepurple},
    basicstyle=\ttfamily\footnotesize,
    breakatwhitespace=false,         
    breaklines=true,                 
    captionpos=b,                    
    keepspaces=true,                 
    numbers=left,                    
    numbersep=5pt,                  
    showspaces=false,                
    showstringspaces=false,
    showtabs=false,                  
    tabsize=2
}
\lstset{style=mystyle}

\usepackage{times}
\usepackage{tikz}
\usetikzlibrary{shapes.multipart, shapes.symbols, arrows.meta, positioning, fit}

\usepackage{sectsty}
\usepackage[compact]{titlesec}

\sectionfont{\normalfont\normalsize\bfseries}
\subsectionfont{\normalfont\normalsize\bfseries}
\subsubsectionfont{\normalfont\normalsize\bfseries}

% Comando para marcar o texto para revisão.
\newcommand{\rev}[1]{\textcolor{red}{#1}}

% Permite escrever aspas normais "text" em vez de ``text''
\usepackage[autostyle]{csquotes}
\MakeOuterQuote{"}


\begin{document}

\begin{titlepage}
	\begin{center}

\begin{figure}[H]
    \centering
    \includegraphics[width=0.5\linewidth]{img/LogoUnesp.png}
\end{figure}

	\vspace{5cm}
    \textbf{\Huge{Survey sobre computação gráfica na física}} \\
    \vspace{5pt}
    Um estudo de caso sobre a história e uma aplicação prática.
        
	\vspace{5cm}
    Carlos Eduardo Nogueira Silva \\
    Felipe Gomes da Silva \\
    Gabriel Martins Brum \\
    Luis Henrique Salomão Lobato \\
	\end{center}
	
	\vspace{1cm}
	\begin{center}
		\vspace{\fill}
    \large{São José do Rio Preto \- SP}\\
    \large{Setembro, 2025} 
	\end{center}
\end{titlepage}

% Table of contents
\tableofcontents

\newpage
\section*{Abstract}

The evolution of computer graphics is inextricably linked to the progressive integration and simulation of physical phenomena. This project presents a comprehensive survey examining the symbiotic relationship between physics and graphical computation, tracing its historical development and impact on the creation of realistic virtual environments. The study first charts the historical trajectory of key advancements, from foundational rendering algorithms to the advent of modern neural rendering techniques. Subsequently, it provides a detailed analysis of foundational simulation domains where physics is paramount, including particle systems for modeling diffuse objects, procedural terrain generation using noise, and the complex dynamics of fluid simulation. Furthermore, to substantiate the surveyed concepts, the paper introduces a practical, educational case study focused on the real-time simulation and rendering of volumetric clouds and noise based terrain generation. This implementation is architected within the Unity engine, leveraging the Universal Render Pipeline (URP). The methodology employs GPU acceleration (GPGPU), utilizing \textit{Compute Shaders} to numerically solve the advection-diffusion equation for fluid density and a custom \textit{volumetric ray marching} shader for physically-based rendering. This work concludes by highlighting the convergence of physical models and computational techniques, demonstrating through its case study the feasibility of implementing complex, optimized simulations in real-time by harnessing contemporary GPU architectures.

\newpage

\input{intro.tex}
\input{terrenos.tex}
\input{sistema-de-particulas.tex}
\input{fluidos.tex}

\section{Geração Volumétrica de Nuvens em tempo real com Unity URP}
Nesta seção, abordaremos a implementação prática de um sistema de geração volumétrica de nuvens em tempo real utilizando o Unity Universal Render Pipeline (URP), apoiando-nos nos conceitos teóricos previamente discutidos. 


\subsection{Objetivos}

Este projeto tem, portanto, um foco educacional ao se utilizar uma ferramenta comum como o Unity e mostrar a viabilidade de se implementar uma simulação complexa e otimizada com GPU em tempo real, conduzindo uma experimentação de técnicas distinas, consonantemente a um aprimoramento academico dos alunos aqui presentes. Ao se desenvolver um sistema de geração volumétrica de nuvens em tempo real, esperamos alcançar os seguintes objetivos específicos:

\begin{itemize}
    \item \textbf{Implementação de técnicas de geração volumétrica:} Desenvolver e implementar técnicas avançadas de geração volumétrica de nuvens, como noise functions (Perlin, Simplex), Worley noise, e outras abordagens baseadas em shaders.
    \item \textbf{Otimização para tempo real:} Adaptar e otimizar os algoritmos de geração volumétrica para garantir que possam ser executados em tempo real dentro do Unity URP, mantendo um equilíbrio entre qualidade visual e desempenho.
    \item \textbf{Integração com Unity URP:} Integrar o sistema de geração volumétrica de nuvens com o Unity Universal Render Pipeline, aproveitando suas capacidades de renderização e iluminação para melhorar a aparência das nuvens.
\end{itemize}

\subsection{Metodologia}
\label{sec:metodologia}

A concretização dos objetivos educacionais deste projeto foi alcançada através da implementação de um sistema de simulação e renderização em tempo real, utilizando o motor Unity sob a arquitetura do \textit{Universal Render Pipeline} (URP). Esta plataforma foi selecionada por sua capacidade de integrar rotinas de computação aceleradas por GPU (GPGPU), um requisito fundamental para processar as simulações complexas em tempo real.

O \textit{pipeline} metodológico foi dividido em duas etapas principais, ambas executadas inteiramente no processador gráfico. A primeira etapa, a "modelagem de forma", consiste na simulação da dinâmica dos fluidos. Para este fim, foram empregados \textit{Compute Shaders} para resolver numericamente as equações que governam o comportamento dos fluidos, como a equação de advecção-difusão, que modela a densidade volumétrica. Esta abordagem permitiu a implementação eficiente dos passos de advecção (pelo método semi-Lagrangiano), difusão e adição de fontes, atualizando um \textit{grid} tridimensional que representa o estado do volume a cada quadro. Podemos ver a interação dos componentes do projeto no seguinte 'diagrama': 

\input{diagrama.tex}

Subsequentemente, a segunda etapa consistiu na renderização deste volume dinâmico. Foi implementada a técnica de \textit{volumetric ray marching}, cuja fundamentação teórica foi apresentada na Seção 5. Um \textit{shader} customizado foi desenvolvido para marchar raios através do \textit{grid} de densidade, que é continuamente atualizado pelo \textit{Compute Shader}. Este \textit{shader} de renderização foi integrado ao URP para amostrar o volume e calcular a interação da luz (absorção e espalhamento), gerando a representação visual final das nuvens volumétricas.


\subsection{Resultados}
\label{sec:resultados}

Os resultados podem ser observados em tempo real na interface do Unity, onde as nuvens volumétricas são renderizadas dinamicamente conforme o sistema de simulação é atualizado. Também, para fins de documentação e análise, deixar-se-á no \href{https://github.com/Bolt-and-plugs/survey-comp-grafica}{repositório} do projeto o link para o slide da apresentação, que inclui capturas de tela e vídeos demonstrativos do sistema em funcionamento\footnote{Link para o slide da apresentação: \url{https://www.canva.com/design/DAG1U0R6q9I/u0j3PF6POgk6FbjRzeiIeQ/view?utm_content=DAG1U0R6q9I&utm_campaign=designshare&utm_medium=link2&utm_source=uniquelinks&utlId=ha7dc0742d2}}


\section{O Impacto do Machine Learning no Desempenho de Simulações Físicas}

Como observado na introdução deste trabalho, os avanços recentes em \emph{Machine Learning} (ML) estão redefinindo as fronteiras da computação gráfica. Em aplicações físicas, o impacto é particularmente transformador, abordando diretamente o principal gargalo dos \emph{solvers} numéricos tradicionais: o alto custo computacional. Mais do que apenas uma ferramenta, o ML introduz novos paradigmas que otimizam o desempenho em três eixos principais: aceleração de simulação, ganhos de fidelidade e memória, e aprimoramento de detalhes\cite{Wang2024}.

\subsection{Aceleração de Simulação (Modelos Surrogados e PINNs)}

A abordagem mais direta para ganhos de desempenho é o uso de redes neurais como \textbf{modelos substitutos} (ou \emph{surrogate models}). A ideia central é que, em vez de executar um \emph{solver} numérico iterativo e custoso a cada quadro --- como a simulação de erosão hidráulica (Seção 3.4) ou a solução das equações de Navier-Stokes (Seção 4.2) ---, uma rede neural é pré-treinada para aprender a função de aproximação desse \emph{solver}.

Uma vez treinada, a rede pode realizar uma inferência (uma única passagem, ou \emph{forward pass}) para prever o resultado da simulação, muitas vezes de forma instantânea. O \emph{survey} de Wang et al. (2024) destaca que Redes Neurais Convolucionais (CNNs) e Redes Neurais Artificiais (ANNs) já são introduzidas para acelerar o passo de projeção de pressão em simulações de fluidos (Figura \ref{fig:ann_pressure}). Outras abordagens usam arquiteturas como LSTMs para prever a evolução temporal de um fluido em um "espaço latente" de baixa dimensão, acelerando drasticamente a simulação\cite{Wang2024}.


Uma segunda abordagem, mencionada na Seção 1.2, são as \textbf{Redes Neurais Informadas pela Física (PINNs)}. Em vez de aprender a partir de dados de simulação, as PINNs aprendem a própria solução da equação diferencial parcial (PDE). A função de perda da rede é o resíduo da própria equação física (ex: a equação de advecção-difusão). A rede neural torna-se, efetivamente, o próprio \emph{solver}.

\begin{figure}[h]
    \centering
    \includegraphics[width=0.8\textwidth]{img/ANN.png}
    \caption{Exemplo de uma Rede Neural Artificial (ANN) usada para acelerar o cálculo de projeção de pressão em uma simulação de fluido. A rede aprende a mapear vetores de características (velocidade, condições de contorno) diretamente para o campo de pressão. Imagem de Wang et al.\cite{Wang2024}.}
    \label{fig:ann_pressure}
\end{figure}

\subsection{Fidelidade Visual e Eficiência de Memória (Renderização Neural)}

Além de acelerar a \emph{simulação}, o ML revolucionou a \emph{renderização}. O exemplo proeminente é o \textbf{Neural Radiance Fields (NeRF)}, como introduzido por Mildenhall et al. (2020)\cite{Mildenhall2020}.

O NeRF substitui representações de cena tradicionais (como malhas de polígonos ou grades de voxels) por uma rede neural (um MLP). Esta rede representa a cena como uma função 5D contínua, que mapeia uma coordenada espacial 3D ($x, y, z$) e uma direção de visualização 2D ($\theta, \phi$) para uma cor ($RGB$) e uma densidade de volume ($\sigma$).

Para gerar uma imagem, o NeRF utiliza a mesma técnica de \textbf{renderização volumétrica clássica} (\emph{ray marching}) que é fundamental para a renderização de nuvens, como a explorada em nosso estudo de caso (Seção 4.3). Ao consultar a rede neural em múltiplos pontos ao longo de cada raio da câmera e acumular os resultados, o NeRF produz renderizações fotorrealistas de novos pontos de vista.

O ganho de desempenho aqui é duplo:
\begin{itemize}
    \item \textbf{Qualidade:} O NeRF alcança resultados de ponta, superando métodos anteriores na síntese de novas visualizações fotorrealistas.
    \item \textbf{Memória:} A abordagem contorna os "custos proibitivos de armazenamento" das grades de voxels discretizadas. Uma cena complexa pode ser armazenada nos pesos da rede, exigindo apenas uma fração do custo de armazenamento (ex: 5MB para os pesos da rede contra 15GB para representações de \emph{voxels} em um dos exemplos).
\end{itemize}

\subsection{Aprimoramento de Detalhes e Controle Artístico}

Finalmente, o ML permite uma abordagem híbrida, onde simulações físicas de baixa resolução (rápidas de calcular) são aprimoradas por redes neurais para gerar detalhes de alta resolução. Esta é uma área de pesquisa ativa.

O \emph{survey} de Wang et al. (2024) classifica isso como \textbf{métodos \emph{data-driven} para aprimoramento de detalhes}. A técnica mais comum é o \emph{upsampling} ou \textbf{super-resolução}. Redes neurais, como as GANs (Generative Adversarial Networks), podem ser treinadas para pegar um campo de velocidade ou densidade de baixa resolução e sintetizar detalhes turbulentos de alta frequência que são temporalmente coerentes \cite{Wang2024}.

Além disso, o ML oferece novos níveis de \textbf{controle artístico}, que são difíceis de alcançar com \emph{solvers} puramente físicos. Isso inclui:
\begin{itemize}
    \item \textbf{Transferência de Estilo (Style Transfer):} Otimização de uma simulação para que ela corresponda a características visuais de uma imagem ou vídeo.
    \item \textbf{Estimativa de Parâmetros:} Uso de redes neurais para extrair parâmetros físicos (como viscosidade) a partir de vídeos do mundo real.
    \item \textbf{Geração por Exemplo:} Uso de GANs para gerar terrenos que seguem o estilo de dados de elevação do mundo real, indo além dos padrões genéricos do \emph{fBm} (Seção 3.2).
\end{itemize}
\section{Conclusão}

Este trabalho realizou um \emph{survey} abrangente sobre a relação simbiótica entre a física e a computação gráfica, demonstrando como a busca pelo realismo visual tem sido historicamente impulsionada pela integração de modelos físicos. A trajetória analisada, que se inicia com os desafios de simulação no MANIAC e passa pelos fundamentos da renderização fisicamente baseada (PBR), evidencia uma convergência clara: a computação gráfica moderna transcendeu a representação geométrica para se tornar um campo de simulação física computacional. Esta convergência, como explorado na Seção 5, é agora drasticamente acelerada pela introdução de paradigmas de \emph{Machine Learning}.

A análise de domínios fundamentais reforçou esta tese. Demonstrou-se que técnicas procedurais, como o \emph{Fractional Brownian Motion (fBm)} para terrenos, embora eficazes, apenas atingem um patamar de realismo convincente quando submetidas a um pós-processamento baseado em simulação física, como a erosão hidráulica. Da mesma forma, a evolução dos sistemas de partículas para modelos hidrodinâmicos complexos, como SPH e métodos Eulerianos, ilustra a necessidade de solucionar equações que governam o mundo natural — processos que, embora precisos, são conhecidos por seu alto custo computacional.

O estudo de caso prático, focado na \textbf{geração de nuvens volumétricas em tempo real no Unity URP}, serviu como a validação central dos conceitos estudados. Atingindo seu objetivo educacional, a implementação demonstrou a viabilidade de se executar simulações complexas em tempo real, aproveitando arquiteturas GPGPU contemporâneas. Este pipeline de duas etapas na GPU, analisado à luz das pesquisas recentes, destaca-se como um precursor direto das abordagens modernas de ML:

\begin{itemize}
    \item \textbf{Simulação (Compute Shader):} A solução numérica da equação de advecção-difusão para modelar a dinâmica do fluido. Esta é precisamente a classe de \emph{solvers} caros que as abordagens modernas de ML, como modelos substitutos (\emph{surrogate models}) e PINNs, buscam acelerar ou substituir \cite{Wang2024}.
    
    \item \textbf{Renderização (Shader Customizado):} A implementação da equação de renderização volumétrica através do \emph{volumetric ray marching}. Esta é a \emph{mesma técnica de renderização clássica} utilizada pelos \emph{Neural Radiance Fields} (NeRFs) para consultar redes neurais (MLPs) que representam a cena, contornando os custos proibitivos de memória de grades de voxels tradicionais \cite{Mildenhall2020}.
\end{itemize}

Este projeto, portanto, conclui que a física não é apenas uma ferramenta acessória, mas um pilar conceitual e estrutural na criação de mundos virtuais dinâmicos. O \emph{Machine Learning} surge não como um substituto para este pilar, mas como um poderoso catalisador: um conjunto de técnicas para acelerar, aprimorar a fidelidade e permitir novos níveis de controle artístico sobre estas simulações físicas \cite{Wang2024}, definindo a nova fronteira do realismo em tempo real.

\newpage
\bibliographystyle{plain}
\bibliography{bibliography}
\end{document}
