\section*{Abstract}

The evolution of computer graphics is inextricably linked to the progressive integration and simulation of physical phenomena. This project presents a comprehensive survey examining the symbiotic relationship between physics and graphical computation, tracing its historical development and impact on the creation of realistic virtual environments. The study first charts the historical trajectory of key advancements, from foundational rendering algorithms to the advent of modern neural rendering techniques. Subsequently, it provides a detailed analysis of foundational simulation domains where physics is paramount, including particle systems for modeling diffuse objects, procedural terrain generation using noise, and the complex dynamics of fluid simulation. Furthermore, to substantiate the surveyed concepts, the paper introduces a practical, educational case study focused on the real-time simulation and rendering of volumetric clouds and noise based terrain generation. This implementation is architected within the Unity engine, leveraging the Universal Render Pipeline (URP). The methodology employs GPU acceleration (GPGPU), utilizing \textit{Compute Shaders} to numerically solve the advection-diffusion equation for fluid density and a custom \textit{volumetric ray marching} shader for physically-based rendering. This work concludes by highlighting the convergence of physical models and computational techniques, demonstrating through its case study the feasibility of implementing complex, optimized simulations in real-time by harnessing contemporary GPU architectures.
