\section{Geração Volumétrica de Nuvens em tempo real com Unity URP}
Nesta seção, abordaremos a implementação prática de um sistema de geração volumétrica de nuvens em tempo real utilizando o Unity Universal Render Pipeline (URP), apoiando-nos nos conceitos teóricos previamente discutidos. 


\subsection{Objetivos}

Este projeto tem, portanto, um foco educacional ao se utilizar uma ferramenta comum como o Unity e mostrar a viabilidade de se implementar uma simulação complexa e otimizada com GPU em tempo real, conduzindo uma experimentação de técnicas distinas, consonantemente a um aprimoramento academico dos alunos aqui presentes. Ao se desenvolver um sistema de geração volumétrica de nuvens em tempo real, esperamos alcançar os seguintes objetivos específicos:

\begin{itemize}
    \item \textbf{Implementação de técnicas de geração volumétrica:} Desenvolver e implementar técnicas avançadas de geração volumétrica de nuvens, como noise functions (Perlin, Simplex), Worley noise, e outras abordagens baseadas em shaders.
    \item \textbf{Otimização para tempo real:} Adaptar e otimizar os algoritmos de geração volumétrica para garantir que possam ser executados em tempo real dentro do Unity URP, mantendo um equilíbrio entre qualidade visual e desempenho.
    \item \textbf{Integração com Unity URP:} Integrar o sistema de geração volumétrica de nuvens com o Unity Universal Render Pipeline, aproveitando suas capacidades de renderização e iluminação para melhorar a aparência das nuvens.
\end{itemize}

\subsection{Metodologia}
\label{sec:metodologia}

A concretização dos objetivos educacionais deste projeto foi alcançada através da implementação de um sistema de simulação e renderização em tempo real, utilizando o motor Unity sob a arquitetura do \textit{Universal Render Pipeline} (URP). Esta plataforma foi selecionada por sua capacidade de integrar rotinas de computação aceleradas por GPU (GPGPU), um requisito fundamental para processar as simulações complexas em tempo real.

O \textit{pipeline} metodológico foi dividido em duas etapas principais, ambas executadas inteiramente no processador gráfico. A primeira etapa, a "modelagem de forma", consiste na simulação da dinâmica dos fluidos. Para este fim, foram empregados \textit{Compute Shaders} para resolver numericamente as equações que governam o comportamento dos fluidos, como a equação de advecção-difusão, que modela a densidade volumétrica. Esta abordagem permitiu a implementação eficiente dos passos de advecção (pelo método semi-Lagrangiano), difusão e adição de fontes, atualizando um \textit{grid} tridimensional que representa o estado do volume a cada quadro. Podemos ver a interação dos componentes do projeto no seguinte 'diagrama': 

\input{diagrama.tex}

Subsequentemente, a segunda etapa consistiu na renderização deste volume dinâmico. Foi implementada a técnica de \textit{volumetric ray marching}, cuja fundamentação teórica foi apresentada na Seção 5. Um \textit{shader} customizado foi desenvolvido para marchar raios através do \textit{grid} de densidade, que é continuamente atualizado pelo \textit{Compute Shader}. Este \textit{shader} de renderização foi integrado ao URP para amostrar o volume e calcular a interação da luz (absorção e espalhamento), gerando a representação visual final das nuvens volumétricas.


\subsection{Resultados}
\label{sec:resultados}

Os resultados podem ser observados em tempo real na interface do Unity, onde as nuvens volumétricas são renderizadas dinamicamente conforme o sistema de simulação é atualizado. Também, para fins de documentação e análise, deixar-se-á no \href{https://github.com/Bolt-and-plugs/survey-comp-grafica}{repositório} do projeto o link para o slide da apresentação, que inclui capturas de tela e vídeos demonstrativos do sistema em funcionamento\footnote{Link para o slide da apresentação: \url{https://www.canva.com/design/DAG1U0R6q9I/u0j3PF6POgk6FbjRzeiIeQ/view?utm_content=DAG1U0R6q9I&utm_campaign=designshare&utm_medium=link2&utm_source=uniquelinks&utlId=ha7dc0742d2}}
